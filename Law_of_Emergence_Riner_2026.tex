\documentclass[12pt,a4paper]{article}

\usepackage[utf8]{inputenc}
\usepackage[T1]{fontenc}
\usepackage{amsmath,amssymb}
\usepackage{geometry}
\usepackage{hyperref}
\usepackage{natbib}
\usepackage{braket}

\geometry{margin=1in}

\hypersetup{
    colorlinks=true,
    linkcolor=blue,
    citecolor=blue,
    urlcolor=blue
}

\title{\textbf{The Law of Emergence: Benford's Distribution as a Universal Constraint on Physical Reality}}
\author{Christopher Riner\\Chesapeake, Virginia\\\texttt{chrisriner45@gmail.com}}
\date{Draft --- February 2026}

\begin{document}

\maketitle

\begin{abstract}
Since 1881, Benford's Law --- the observation that the leading digits of naturally
occurring numerical datasets follow a logarithmic distribution $P(d) = \log_{10}(1 + 1/d)$ ---
has been treated as a statistical curiosity, a forensic tool, and an unexplained
regularity. This paper proposes a fundamental reinterpretation: that Benford's
distribution is not a consequence of physical law but a precondition for it --- the
single axiom from which physical law emerges. We propose that Einstein's discovery
of light's constancy, far from being a separate invariant, was the identification
of the purest physical expression of this deeper constraint: light, having no mass
and therefore no entropy, conforms to the logarithmic distribution perfectly, and
its constancy is what zero deviation from the axiom looks like. Mass introduces
deviation; gravity is the constraint's self-regulatory response; the fundamental
forces are the constraint expressed through specific degrees of freedom. We examine
existing empirical evidence across atomic, subatomic, and quantum scales; explore
connections to entropic gravity, gravitational time dilation, and the
quantum-classical boundary; and propose that this distributional constraint is
logically prior to the universe, requiring neither spacetime, matter, nor energy to
exist. A research program is outlined for systematically measuring the outputs of
physical equations against the Benford baseline, with particular attention to the
role of optical computing architectures --- which operate natively in the logarithmic
domain --- as instruments uniquely suited to this inquiry.
\end{abstract}

\newpage
\tableofcontents
\newpage

%% ============================================================
\section{Introduction}
%% ============================================================

Physics possesses one established universal invariant: the speed of light. Einstein's
central insight was not the discovery of this constancy --- Michelson and Morley had
already measured it --- but the decision to treat it as axiomatic. By accepting that
light's speed does not change and allowing everything else to adjust around it,
Einstein derived special relativity (1905), general relativity (1915), and
fundamentally reshaped our understanding of space, time, mass, and energy.

This paper proposes that light's constancy was not the axiom itself but its most
visible consequence --- and that the true axiom has been hiding in plain sight for
145 years.

Benford's Law, first observed by Simon Newcomb in 1881 \cite{newcomb1881} and empirically validated
by Frank Benford in 1938 \cite{benford1938}, describes the logarithmic distribution of leading
digits in naturally occurring datasets. The probability of a first significant digit
$d$ is given by:

\begin{equation}
P(d) = \log_{10}\!\left(1 + \frac{1}{d}\right)
\end{equation}

This distribution has been confirmed across an extraordinary range of domains:
physical constants \cite{burke1991}, nuclear decay half-lives across all three non-gravitational
forces \cite{ni2008}, hadron properties \cite{shao2009}, Bose-Einstein statistics \cite{shao2010}, atomic spectra \cite{ralchenko2024},
river areas, population figures, astronomical distances \cite{alexopoulos2014}, molecular weights, and
financial data, among many others.

Despite extensive empirical confirmation and multiple partial theoretical
explanations --- scale invariance \cite{pinkham1961}, base invariance \cite{hill1995base}, central-limit-type mixing
\cite{hill1995statistical}, maximum entropy \cite{kafri2009,lemons2019}, and Markov convergence \cite{burgos2021} --- no single derivation
explains all instances from first principles \cite{berger2011}. The question of \emph{why} Benford's
Law appears universally remains open.

We propose that the answer requires an inversion deeper than even Einstein's.
Einstein elevated light's constancy from observation to axiom. We propose elevating
Benford's distribution to the sole axiom --- and repositioning light's constancy as
its first and purest consequence. Benford's distribution is not a pattern that
emerges from physics. It is the constraint that physics must satisfy in order to
emerge at all. Light's constancy is what that constraint looks like when mass is
zero.

%% ============================================================
\section{One Axiom}
%% ============================================================

\subsection{The Constraint}

Benford's Law appears in datasets generated by every known physical mechanism:
strong nuclear force processes, weak force processes, electromagnetic processes,
thermodynamic processes, quantum statistical distributions, astrophysical phenomena,
geological phenomena, and biological phenomena. It holds across scales from
subatomic particles to galaxy clusters. It is scale-invariant \cite{pinkham1961}, base-invariant
\cite{hill1995base}, and has been shown to be an attractor state analogous to thermodynamic
equilibrium \cite{burgos2021}.

This universality is currently treated as something to be explained rather than
something to build upon. Multiple theoretical frameworks each account for subsets
of its appearances, but none unifies them. We propose that this is because the
explanations are pointing in the wrong direction. Benford's distribution does not
emerge from the specific mechanisms of physics. It is the constraint that those
mechanisms must satisfy.

Benford's distribution is not a physical object. It is a mathematical constraint ---
massless, timeless, and not subject to any physical process. Like the statement
``$2 + 2 = 4$,'' it does not depend on the existence of a physical medium. It is the
one axiom from which everything else follows.

\subsection{Light as Perfect Conformance}

Before Einstein, the constancy of the speed of light was a problem. It contradicted
Galilean relativity. It violated the intuition that velocities should be additive.
Physicists attempted to explain it away --- the luminiferous aether, Lorentz
contraction, various compensatory mechanisms.

Einstein's contribution was to stop explaining and start building. He accepted $c$ as
invariant and asked: if light doesn't change, what does? The answer was everything
else. Space contracts. Time dilates. Mass increases with velocity. Energy and mass
are interconvertible. The invariant became the foundation, and the rest of physics
reorganized around it.

Einstein identified the most visible expression of the deeper axiom. But light's
constancy is not itself the axiom. It is the \textbf{first and purest consequence} of
the logarithmic constraint.

Light has no mass. No mass means no entropy. No entropy means no deviation from
the constraint. A photon emitted at the edge of the observable universe 13.8
billion years ago arrives unchanged --- redshifted by the expansion of space (a
property of the medium, not the photon), but intrinsically identical. It
experienced no time, no decay, no disorder.

Light is constant because it conforms to the logarithmic constraint perfectly.
With zero mass, there is nothing to introduce deviation. Light does not stand
alongside the constraint as a co-equal axiom. It is what the constraint looks
like when nothing gets in the way.

\subsection{Mass as Deviation}

If light --- with zero mass --- represents perfect conformance with the logarithmic
constraint, then mass represents deviation from it. The more mass a system has,
the more entropy it is subject to, and the more its behavior deviates from the
constraint's baseline.

This produces a hierarchy:

\begin{itemize}
\item \textbf{The logarithmic constraint} --- the one axiom
\item \textbf{Zero Mass} --- One example: Light, zero deviation, perfect conformance (constant speed, no time experienced, no entropy)
\item \textbf{Gravity} --- the self-regulatory response to mass-induced deviation; the constraint correcting itself through entropic feedback (Section~\ref{sec:gravity})
\item \textbf{The three domain-specific forces} --- the constraint expressed through specific degrees of freedom (charge, color, flavor) at scales where mass is present
\item \textbf{Matter, time, entropy} --- the flexing of systems that deviate from perfect conformance, all measured against the one rule they must satisfy
\end{itemize}

Einstein discovered the purest physical expression of the axiom and built a
century of physics from it. But he was observing a consequence, not the cause.
The cause is the logarithmic constraint. Light's constancy is what that cause
looks like when mass is zero. Everything else --- space, time, gravity, the
forces --- is what it looks like when mass is not zero and the universe must
flex to accommodate deviation from the one rule.

\subsection{The Speed of Light Barrier: The Axiom Says No}

Standard physics states that no object with mass can reach the speed of light
because its relativistic mass would become infinite, requiring infinite energy.
This is accurate as a description. But it describes the symptom, not the cause.

In this framework, the cause is simpler: mass carries deviation from the
logarithmic constraint. Light carries zero deviation. If an object with mass
reached the speed of light, a thing with deviation would be behaving as though
it had zero deviation. This is a contradiction of the axiom. It is not that the
universe cannot supply enough energy. It is that the constraint will not permit
a deviation-carrying entity to behave as though it carries none.

The energy requirement approaching infinity is not the cause of the barrier. It
is the \textbf{enforcement mechanism} --- the way the axiom says no.

This reframes the relativistic speed limit as one instance of a general
principle: the logarithmic constraint enforcing itself. Multiple phenomena in
physics that appear to be distinct laws are, in this framework, different
expressions of the same enforcement:

\begin{itemize}
\item \textbf{The speed of light barrier} --- the axiom preventing mass from mimicking zero deviation
\item \textbf{Gravitational time dilation} --- the axiom throttling its own rate where deviation concentrates (Section~\ref{sec:timedilation})
\item \textbf{The black hole event horizon} --- the axiom containing runaway deviation by sealing the geometry (Section~\ref{sec:blackholes})
\item \textbf{Accelerating expansion in voids} --- the axiom unthrottled where deviation is absent (Section~\ref{sec:darkenergy})
\end{itemize}

Four phenomena. Four enforcement mechanisms. One rule.

\subsection{Time as Perceived Entropy}

Every method of measuring time is, at root, a measurement of entropy. A clock
ticks because a mechanism transitions between states --- entropy. A candle burns
down --- entropy. Biological organisms age because cellular processes accumulate
irreversible changes --- entropy. Memories form because neural configurations
change irreversibly --- entropy. The ``arrow of time'' has long been recognized as
the arrow of entropy. Physicists have treated the two as correlated. This
framework proposes they are identical.

\textbf{Time is the word we use for what it feels like to experience entropy from
inside a system made of mass.}

This resolves several otherwise puzzling features of time:

\begin{itemize}
\item \textbf{A photon experiences no time.} Not because time ``stops'' for light, but
  because light has no mass, no entropy, and no deviation. There is nothing
  to experience. Time is not paused for the photon. Time was never there.
  Time is a property of deviation, and light has none.

\item \textbf{Time slows in a gravitational field.} Not because spacetime geometry acts
  on time as a separate entity, but because the logarithmic constraint throttles
  entropy in regions of concentrated deviation (Section~\ref{sec:timedilation}). We perceive that
  throttling as time slowing because, to a mass-bearing observer, entropy and
  time are indistinguishable.

\item \textbf{Time had a beginning (the Big Bang).} If time is entropy, then ``the
  beginning of time'' is simply the beginning of entropy --- the moment emergence
  first occurred under the logarithmic constraint. The constraint itself,
  being logically prior to the universe (Section~\ref{sec:bigbang}), does not require time.
  Time began when the constraint first had mass to act upon.

\item \textbf{Time flows in one direction.} The second law of thermodynamics states
  that entropy increases. If time is perceived entropy, its unidirectionality
  is not a mystery requiring explanation. It is simply what increasing entropy
  feels like from the inside.
\end{itemize}

\subsection{Entropy as Return to Conformance}

The standard interpretation of entropy is disorder --- systems decay, structures
dissolve, complexity breaks down. This framing is descriptive but directionless.
It characterizes entropy as purposeless degradation.

This framework reinterprets entropy as \textbf{mass attempting to return to zero
deviation} --- to the massless state of perfect conformance with the logarithmic
constraint. Mass is deviation. Deviation is inherently unstable under the
constraint. The axiom does not allow it to remain static. It must move, and the
only direction available is back toward conformance.

This reinterpretation is supported by the actual behavior of entropic processes:

\begin{itemize}
\item \textbf{Stars} burn through fuel, converting mass to energy, radiating photons and
  other massless particles --- shedding mass, moving toward zero deviation
\item \textbf{Radioactive decay} transforms unstable nuclei into more stable, lower-energy
  configurations --- reducing mass-energy, approaching conformance
\item \textbf{Thermal equilibrium} distributes energy uniformly --- spreading deviation as
  thinly as possible across the system
\item \textbf{Black hole evaporation} (Hawking radiation) converts the most extreme
  concentration of mass back into radiation over $\sim 10^{67}$ years --- the most
  dramatic return from maximum deviation to the massless state
\item \textbf{The heat death of the universe} --- the theoretical end state of maximum
  entropy --- is a universe of maximally dispersed, minimally interacting
  particles and low-energy radiation. As close to massless, and therefore as
  close to zero deviation, as the universe can achieve
\end{itemize}

The end state of entropy is not disorder. It is the universe approaching
conformance with the axiom as completely as mass allows.

This explains why anything with mass has entropy: \textbf{mass is deviation, and the
constraint does not permit deviation to be stable.} The axiom continuously
drives mass-bearing systems toward the massless state --- toward zero deviation,
toward perfect conformance. All massless entities travel at $c$; they experience
no time, no decay, no entropy. That is the state the constraint favors. Everything
with mass is being moved toward it.

We call that movement entropy. We perceive it as time (Section~2.5). Its
directionality --- the arrow of time, the second law of thermodynamics --- is not
a mystery. It is the axiom enforcing return to conformance. Entropy has a
direction because the constraint has a preferred state: massless, zero deviation,
perfect conformance.

\subsection{Spacetime as Description}

General relativity describes spacetime as a dynamic fabric --- curved by mass,
warped by energy, inseparable from the matter within it. This description is
the most successful in the history of physics.

But spacetime, in this framework, is not a fundamental entity. It is a
\textbf{description of the effects} of the logarithmic constraint operating on mass.
Einstein provided the map --- the most precise and predictive map ever drawn.
The territory being mapped is the axiom itself.

Spacetime is what the constraint's effects look like from the inside, when the
observer is made of mass and is experiencing deviation. Space curves because
deviation distorts the local geometry. Time dilates because the constraint
self-regulates in regions of concentrated deviation. The fabric of spacetime
is the fabric of the axiom's enforcement, perceived from within.

%% ============================================================
\section{The Logarithmic Constraint and Emergence}
%% ============================================================

\subsection{Emergence as the Central Concept}

The known laws of physics are descriptive. They characterize the behavior of systems
that already exist. Newton's laws describe motion. Maxwell's equations describe
electromagnetic fields. The Schr\"{o}dinger equation describes quantum state evolution.
None of them address the question of why organized systems exist in the first place ---
why there is structure rather than noise.

We propose that Benford's logarithmic distribution addresses this gap. It is not a
law describing behavior within an existing system. It is a \textbf{condition} that must be
satisfied for a system to emerge and sustain itself.

\subsection{The Constraint on Existence}

If Benford's distribution is a precondition for emergence, its universality follows
immediately: every organized system that exists conforms to it, not because the
systems communicate or share mechanisms, but because any configuration that failed to
conform could not sustain itself into existence. This is survivorship at the deepest
possible level.

This framing transforms Benford's Law from an observation requiring explanation into
an axiom generating predictions.

\subsection{The Logarithm as Structural Necessity}

The appearance of the logarithm in this constraint is supported by independent
results from multiple fields:

\begin{itemize}
\item \textbf{Information theory:} Shannon (1948) proved that the logarithm is the \emph{unique}
  function satisfying the axioms of information measurement \cite{shannon1948}. It is not a
  convention. It is a mathematical necessity.
\item \textbf{Statistical mechanics:} Boltzmann's entropy $S = k \ln W$ requires the logarithm
  to ensure additivity for independent systems \cite{boltzmann1877}.
\item \textbf{Maximum entropy:} Jaynes (1957) showed that the logarithmic form of entropy is
  a logical necessity --- the only function satisfying consistency, additivity, and
  continuity \cite{jaynes1957}.
\item \textbf{Scale transformations:} The renormalization group, which governs how physics
  changes across scales, is parameterized logarithmically \cite{wilson1971}. The logarithm is
  the natural coordinate of scale.
\end{itemize}

These independent derivations, from different fields and different decades, all
converge on the same conclusion: the logarithm is not one option among many. It is
the unique mathematical structure that bridges multiplicative and additive processes,
governs the flow of information, and parameterizes the relationship between scales.

If reality has a native mathematical language, these results collectively indicate
that it is logarithmic.

%% ============================================================
\section{Gravity as the Braking Mechanism of Emergence}
\label{sec:gravity}
%% ============================================================

\subsection{The Anomaly of Gravity}

Among the four fundamental forces, gravity is anomalous in several respects:

\begin{itemize}
\item It is approximately $10^{36}$ times weaker than the electromagnetic force
\item It cannot be quantized within existing frameworks
\item Unlike the other three forces, which act on specific charges (electric charge,
  color charge, flavor), gravity acts on \textbf{everything} with mass-energy
\end{itemize}

This universality parallels Benford's distribution, which also appears across all
domains without restriction to specific mechanisms or interactions.

\subsection{Entropic Gravity}

Verlinde (2011) proposed that gravity is not a fundamental force but an emergent
entropic phenomenon \cite{verlinde2011}, building on Jacobson's (1995) derivation of Einstein's
field equations from thermodynamic entropy \cite{jacobson1995}. In this framework, gravitational
attraction is what it looks like when matter moves toward configurations of higher
entropy --- analogous to how a polymer contracts not through a ``contraction force'' but
through statistical mechanics favoring higher-entropy configurations.

\subsection{Self-Regulation Through Time Dilation}
\label{sec:timedilation}

If gravity is an entropic phenomenon, and gravitational fields produce time dilation
(experimentally confirmed and operationally corrected for in GPS systems daily), then
a remarkable self-regulatory mechanism emerges:

\begin{enumerate}
\item Entropy concentrates in regions of high mass-energy density
\item This concentration produces gravitational effects (Verlinde)
\item These gravitational effects slow local time (general relativistic time dilation)
\item Slowed time reduces the local rate of entropy change
\item The system reaches equilibrium
\end{enumerate}

\textbf{Entropy slows itself down in regions where it is concentrated.} This is
self-regulation in a single step. The braking mechanism is intrinsic.

This may explain the stability of gravitational systems: why planets maintain orbits,
why galaxies hold together, and why the universe neither flies apart nor collapses
instantly. The entropic feedback loop provides a natural stabilization mechanism.

\subsection{Black Holes: Entropy Containing Light}
\label{sec:blackholes}

At the extreme limit of this self-regulation, entropy concentrates so densely that
it completely blocks its own progression. Time stops at the event horizon. The
self-regulation has reached totality.

This provides a mechanism for why the geometry of spacetime closes at a black hole's
event horizon. The standard account --- that spacetime is so curved that all paths lead
inward --- is descriptively accurate but does not explain \emph{why} the geometry closes.
The entropic self-interference framework offers a causal mechanism: entropy,
by getting in its own way, seals the geometry shut.

Light, having no mass, is not acted upon by gravity or entropy. Nothing pulls it in.
Rather, the geometry in which it travels has been closed by entropy's
self-interference. Light continues traveling at $c$ along straight paths through
the local geometry, but the geometry now has no exit. The container formed around it.

\textbf{The only thing that can trap something immune to entropy is a physical containment
created by entropy obstructing itself.}

This framework also provides a natural explanation for the holographic property of
black holes. The Bekenstein-Hawking result \cite{bousso2002,hawking1975} --- that a black hole's entropy is
proportional to the surface area of its event horizon, not its interior volume --- has
been one of the deepest puzzles in theoretical physics. For ordinary systems, entropy
scales with volume. For black holes, all the information one would expect to require
a volume to encode is fully captured on the two-dimensional boundary. The holographic
principle \cite{bousso2002} generalizes this: the complete description of a volume of space can be
encoded on its boundary surface.

In the entropic self-interference framework, this is expected. The event horizon is
where entropy's self-regulation reaches totality --- where the constraint seals the
geometry shut. The boundary is where the enforcement is happening. The interior is
simply the sealed room. The information is encoded on the surface because the surface
is the seal --- the physical expression of entropy obstructing itself. Notably, Verlinde's
entropic gravity framework \cite{verlinde2011} already employs holographic screens as a foundational
element, indicating that the holographic principle is not an addition to this framework
but structurally inherent in it.

The constraint remains intact inside the black hole. Light still travels at $c$ ---
it does not deviate from the axiom, because it has no mass. Benford's distribution
presumably still holds. The axiom did not break. The room it governs simply has
no door.

\subsection{Dark Energy as Absence of Braking}
\label{sec:darkenergy}

In the voids between galaxy clusters --- regions of minimal mass, minimal entropy
concentration, and minimal gravitational time dilation --- time runs at its fastest
and emergence is unthrottled. These voids are also where the expansion of the
universe accelerates most.

If gravity is the braking mechanism of emergence, then the accelerating expansion
in low-density regions does not require a new form of energy (dark energy) as its
cause. It may simply be what unbraked emergence looks like. Nothing is pushing the
voids apart. Nothing is slowing them down.

\subsection{Why Gravity Appears Weak}

In this framework, gravity's apparent weakness relative to the other forces is not a
fine-tuning problem. Gravity is not weak. It operates at a different level entirely.
The strong, weak, and electromagnetic forces are interactions between particles within
the emergent system. Gravity is the self-regulation of the underlying constraint ---
the system governing its own rate of emergence. Comparing gravity's strength to
electromagnetism is a category error, like comparing the brightness of a lamp to the
voltage of the power grid that supplies it.

%% ============================================================
\section{Across Scales: Physical Laws Measured Against the Constraint}
%% ============================================================

The following survey reframes existing empirical results from the perspective of
this paper's thesis. In each case, the original researchers applied Benford's law
to their data as a test. We reinterpret these results directionally: the data from
each physical domain was measured against the logarithmic constraint, and in every
case, the domain's equations produced outputs that conformed. The constraint held.
The physics satisfied it.

\subsection{Macro Scale}

Benford's original 1938 study measured 20 datasets --- river areas, populations,
physical constants, molecular weights, and more --- against the logarithmic
distribution \cite{benford1938}. All conformed. Subsequent work brought financial data, election
statistics, genomic data, and geophysical measurements to the same constraint \cite{sambridge2010}.
All satisfied it.

\subsection{Atomic Scale}

Ralchenko and Pain (2024) brought NIST atomic spectral data --- line energies,
oscillator strengths, Einstein coefficients, and radiative opacities --- to the
Benford baseline \cite{ralchenko2024}. The atomic equations produced outputs consistent with the
constraint. Burke and Kincanon (1991) measured fundamental physical constants
against the distribution \cite{burke1991}. They conformed.

\subsection{Nuclear and Subatomic Scale}

Ni and Ren (2008) brought 3,177 nuclide half-lives to the logarithmic constraint ---
spanning alpha decay (strong force), beta decay (weak force), and spontaneous
fission (electromagnetic force) \cite{ni2008}. All three forces produced outputs satisfying
the distribution. This is significant: the equations governing three independent
fundamental interactions, when measured against Benford's constraint, all conform.
The constraint does not belong to any single force. The forces belong to it.

Shao and Ma (2009) brought hadron full widths and lifetimes to the same baseline \cite{shao2009}.
The particle physics data satisfied the constraint.

\subsection{Quantum Statistical Mechanics}

Shao and Ma (2010) brought the three fundamental statistical distributions of
physics --- Boltzmann-Gibbs, Fermi-Dirac, and Bose-Einstein --- to the Benford
baseline \cite{shao2010}. The Bose-Einstein distribution satisfies the constraint \textbf{exactly at
all temperatures}. The Boltzmann-Gibbs and Fermi-Dirac distributions show slight
periodic deviations. The authors concluded that Benford's law ``might be a more
fundamental principle behind the complexity of nature'' --- an interpretation consistent
with the thesis proposed here.

\subsection{Quantum Phase Transitions}

Sen(De) and Sen (2011) used the Benford constraint as a diagnostic instrument,
measuring magnetization and correlation data from quantum many-body systems against
it \cite{sende2011}. Deviations from the constraint detected quantum phase transitions --- the
boundary where quantum behavior gives way to classical behavior. Rane et al.\ (2014)
showed that measuring quantum XY model data against the Benford baseline provides
superior finite-size scaling exponents compared to conventional quantum methods \cite{rane2014}.
The constraint, used as the instrument, outperformed the domain's own tools.

\subsection{Astrophysical Scale}

Alexopoulos and Leontsinis (2014) brought galaxy distances, star distances, and
gamma-ray burst properties to the logarithmic constraint \cite{alexopoulos2014}. Astrophysical data
at cosmological scales satisfied it.

\subsection{Summary}

At every scale tested --- from quantum statistical distributions to galaxy clusters,
across all known fundamental forces, and in data generated by every major branch of
physics --- the equations of each domain produce outputs that satisfy Benford's
logarithmic constraint. No domain-specific mechanism explains this universality.
A domain-independent constraint, against which all physics is measured, does.

%% ============================================================
\section{The Quantum-Classical Bridge}
%% ============================================================

\subsection{The Unification Problem}

Physics currently operates with two incompatible frameworks: quantum mechanics for
the very small and general relativity for the very large. Every unification attempt ---
string theory, loop quantum gravity, causal set theory --- has proceeded top-down,
starting with the mathematical structures of both theories and attempting to reconcile
them. Despite decades of effort, no approach has produced experimentally confirmed
predictions.

\subsection{A Bottom-Up Alternative}

The approach proposed here is empirical and bottom-up. Rather than starting with
theory and searching for consistency, we start with Benford's logarithmic constraint
as the fixed reference and bring the equations of each physical domain to it,
tracing downward through every scale:

\begin{enumerate}
\item \textbf{Measure} the outputs of atomic, subatomic, and quantum equations against the
   Benford constraint (partially accomplished --- see Section~5)
\item \textbf{Characterize} where and how each domain's equations deviate from the constraint
\item \textbf{Map} the quantum-to-classical boundary (decoherence) by identifying where
   quantum equations transition from deviation to conformance with the Benford
   baseline
\item \textbf{Compare} how gravitational equations conform to the constraint versus the
   equations of the other three forces
\end{enumerate}

If the decoherence boundary is identifiable as the transition point where quantum
equations begin satisfying the Benford constraint --- the point where emergence
``switches on'' --- it would provide an empirical signature of the quantum-classical
interface derived from the constraint itself, something no current framework offers.

\subsection{String Theory as Inventory}

String theory has produced decades of sophisticated mathematical machinery: extra
dimensions, branes, dualities, conformal field theories, holographic principles, and
the AdS/CFT correspondence. The limitation has not been the quality of the parts but
the absence of an organizing principle connecting them to observable reality ---
resulting in a landscape of approximately $10^{500}$ possible solutions with no method
of selection.

If Benford's distribution is the underlying constraint governing emergence, it may
provide precisely this organizing principle: a criterion that valid physical
configurations must satisfy, reducing the landscape to those solutions consistent
with the logarithmic constraint. The mathematical machinery of string theory would
then not be wasted but rearranged --- existing parts assembled under a new
architecture.

%% ============================================================
\section{Before the Big Bang}
\label{sec:bigbang}
%% ============================================================

Every current framework in physics encounters a boundary at the Big Bang. General
relativity produces singularities. Quantum mechanics requires pre-existing spacetime.
The Planck epoch (the first $10^{-43}$ seconds) remains inaccessible.

A distributional constraint, however, is not subject to these limitations. It does
not require spacetime, matter, or energy to exist. It governs the relationship
between magnitudes, not magnitudes themselves. Like mathematical truths generally, it
is logically independent of the physical universe.

If emergence requires this constraint --- if organized systems can only arise under its
governance --- then the constraint is logically prior to any specific universe. It does
not describe what happens within the universe. It describes the condition under which
a universe can arise at all.

The question ``Did Benford's distribution exist before the Big Bang?'' is therefore not
metaphysical. It is structural. If the constraint is a necessary condition for
emergence, and the universe is an emergent system, then the constraint precedes the
system it enabled.

\subsection{The Constraint as a Filter Across the Boundary}

Einstein's field equations produce families of solutions. Some describe observed
phenomena --- black holes, gravitational waves, the expansion of the universe.
Others describe phenomena never observed in our universe --- most notably white
holes, the time-reversed solutions of black holes. White holes are mathematically
valid under general relativity but have no observational confirmation in our
universe. Similarly, string theory produces approximately $10^{500}$ solutions, all
mathematically consistent, with no method to determine which correspond to
physical reality.

The logarithmic constraint provides a selection criterion. Bring all solutions ---
from general relativity, from string theory, from any mathematical framework ---
to the Benford baseline. Solutions whose outputs conform to the constraint
correspond to physical reality. Solutions that do not conform are eliminated.
This is not a subjective elegance argument or an anthropic selection. It is an
objective, measurable test.

Critically, ``physical reality'' in this framework is not limited to our current
universe. The constraint is logically prior to the Big Bang (Section~\ref{sec:bigbang}). It
does not depend on spacetime. It operates on both sides of the boundary.

This means: if white hole solutions fail to conform to the Benford constraint,
they are unphysical --- they never occurred anywhere, in any regime. But if white
hole solutions \textbf{do} conform to the constraint while failing to describe our
current universe, the implication is that they are physically real but belong to
a different regime --- a pre-Big Bang regime. They satisfy the axiom. They
happened. They are simply not happening here.

The constraint can see what Einstein's equations cannot: past the Big Bang. The
field equations hit a singularity at the boundary. Quantum mechanics requires
spacetime that does not yet exist. But the logarithmic constraint is independent
of spacetime. It was present on both sides of the transition. Solutions that
satisfy it are real, regardless of which side they belong to.

\subsection{The Big Bang as Phase Transition}

If the constraint operated before the Big Bang in a regime described by different
solutions (white holes, imaginary time, reversed entropy), then the Big Bang
itself is a \textbf{phase transition} --- a flip from one regime of the constraint to
its mirror.

This connects to existing physics:

\begin{itemize}
\item \textbf{Hawking and Hartle's no-boundary proposal} describes the early universe
  using imaginary time, where the distinction between time and space disappears.
  In this framework, imaginary time corresponds to the pre-Big Bang regime ---
  the constraint operating before entropy established a direction.
\item \textbf{White holes} are time-reversed black holes. In a pre-Big Bang regime where
  entropy runs in the opposite direction (or has not yet selected a direction),
  white holes would be the natural expression of extreme entropic concentration
  --- the mirror of black holes in our regime.
\item \textbf{The Big Bang} is the transition point. The flip from imaginary to real.
  From white holes to black holes. From undirected to directed entropy. The
  constraint did not change. The regime did.
\end{itemize}

Sen(De) and Sen (2011) demonstrated that Benford's law detects phase transitions
in quantum systems \cite{sende2011}. If the Big Bang is a phase transition between two
regimes of the same constraint, Benford analysis may be capable of
characterizing the transition itself --- identifying the signature of the flip
in the mathematical structure of the solutions on either side.

The universe before ours was not nothing. It was the other side of the axiom.
Same constraint. Different regime. The Big Bang was not the beginning of
reality. It was the point where the constraint changed how it expressed itself.

%% ============================================================
\section{The Simple Formula}
%% ============================================================

Einstein spent the last thirty years of his life searching for a unified field
theory --- a compact formulation unifying all fundamental interactions. Every subsequent
attempt has increased in complexity: more dimensions, more symmetry groups, more
mathematical apparatus.

Einstein believed the answer would be elegant. ``As simple as possible, but no
simpler.''

Benford's Law is:

\begin{equation}
P(d) = \log_{10}\!\left(1 + \frac{1}{d}\right)
\end{equation}

It fits on a napkin. It has been known since 1881. If this distribution is the
precondition for emergence --- the one rule all organized systems must satisfy to
exist --- then it does not compete with the fundamental forces. It generates them.
Gravity, electromagnetism, the strong and weak nuclear forces would each be specific
expressions of this constraint operating through different degrees of freedom at
different scales.

This would be precisely what Einstein sought: not a larger equation containing all
smaller equations, but a simpler rule making all smaller equations inevitable.

%% ============================================================
\section{Proposed Research Program}
%% ============================================================

\subsection{Systematic Measurement of Physical Data Against the Constraint}

Bring the outputs of all available physical measurement databases to the Benford
logarithmic baseline, treating the constraint as the fixed reference and each
domain's data as the variable being measured. Specific targets include:

\begin{itemize}
\item Complete periodic table: measure all properties of all elements against the
  constraint --- identify which atomic equations produce Benford-conformant outputs
  and characterize any deviations
\item Particle Data Group: bring all measurable particle properties to the Benford
  baseline --- determine how the Standard Model's equations satisfy the constraint
\item NIST atomic spectral databases: extend Ralchenko and Pain (2024) by measuring
  the full spectral dataset against the constraint at higher resolution
\item Nuclear decay databases: bring all decay modes to the constraint --- compare how
  strong, weak, and electromagnetic processes each satisfy it
\item Quantum correlation measurements and entanglement data: bring quantum
  observables to the Benford baseline to characterize pre-decoherence deviation
  patterns
\end{itemize}

\subsection{Decoherence Boundary Investigation: Benford's Law as the Instrument}

Consistent with the central thesis of this paper --- that Benford's distribution is
the foundational constraint, not a secondary pattern to be checked for --- the
investigation of the quantum-to-classical boundary should begin from Benford's law
and bring the equations of quantum mechanics to it, rather than the reverse.

The standard equations of quantum mechanics --- the Schr\"{o}dinger equation, the Born
rule ($|\psi|^2 \to$ probability), and the density matrix formalism ($\rho = \ket{\psi}\bra{\psi}$) --- describe
how quantum systems evolve and how probabilities emerge from wavefunctions. The
proposed approach is to re-express these formalisms within Benford's logarithmic
framework and ask what the quantum-to-classical transition looks like from inside
the constraint:

\begin{itemize}
\item \textbf{Reframe quantum probabilities in Benford space.} Take the Born rule outputs
  ($|\psi|^2$) for known quantum systems and map their leading-digit distributions
  against $P(d) = \log_{10}(1 + 1/d)$. Rather than asking ``does this quantum data
  happen to follow Benford's law,'' treat the logarithmic distribution as the
  expected baseline and characterize deviations from it. The deviations become
  the signal.

\item \textbf{Track the deviation through decoherence.} As a quantum system decoheres ---
  transitioning from coherent superposition to classical mixture --- monitor how
  its deviation from Benford conformance evolves. If Benford's distribution is
  the constraint governing classical emergence, then decoherence should manifest
  as a convergence toward the Benford baseline. The decoherence boundary would
  be identifiable as the point where deviation resolves into conformance.

\item \textbf{Express decoherence rates in logarithmic terms.} The standard decoherence
  rate equations involve exponential decay of off-diagonal density matrix elements.
  Re-expressed logarithmically, these decay rates may reveal structure that is
  hidden in the linear formalism --- structure that connects directly to the
  Benford constraint.

\item \textbf{Use Benford deviation as a diagnostic.} If the logarithmic distribution is
  fundamental, then the magnitude and character of a quantum system's deviation
  from Benford conformance would encode information about how far that system is
  from classical emergence. This would provide a new, Benford-native metric for
  ``how quantum'' a system is --- complementary to existing measures like quantum
  discord and entanglement entropy, but derived from the proposed foundational
  constraint rather than from quantum theory itself.
\end{itemize}

The key methodological distinction is directional: Einstein did not test whether
light was constant under relativistic conditions. He assumed constancy and derived
the conditions. Similarly, this program assumes Benford's distribution holds as
the baseline of emergence and uses it to derive the structure of the
quantum-to-classical transition.

\subsection{Gravity Measured Against the Constraint}

Bring gravitational data to the Benford baseline and compare how gravitational
equations satisfy the constraint versus the equations of the other three forces.
If gravity is the self-regulatory expression of the logarithmic constraint (Section~\ref{sec:gravity}),
it should show a distinct conformance signature. Specific investigations:

\begin{itemize}
\item Bring gravitational wave data (LIGO/Virgo) to the Benford baseline --- does the
  constraint reveal structure in gravitational wave signals that linear analysis
  does not?
\item Compare gravitational conformance to electromagnetic, strong, and weak force
  conformance --- does gravity satisfy the constraint differently, more strongly,
  or more fundamentally than the domain-specific forces?
\item Bring mixed-force datasets to the Benford baseline --- can the constraint itself
  distinguish gravitational contributions from those of other forces, acting as
  a separation tool?
\end{itemize}

\subsection{Where Physical Systems Deviate from the Constraint}

Identify conditions under which physical equations produce outputs that deviate from
the Benford baseline. If it is a universal constraint, the nature of each domain's
deviation is as informative as its conformance --- analogous to how a medium's
refractive index (its deviation from light's vacuum speed) reveals the properties
of that medium. The constraint remains fixed. The deviations characterize the
physics.

\subsection{Computational Tools}

The N-Radix wavelength-division ternary optical accelerator \cite{riner2025} --- which operates
natively in the logarithmic domain through wavelength-encoded ternary logic and
sum-frequency generation --- represents a computing architecture uniquely suited to
this investigation. Its native operation in the log domain means it computes in the
same mathematical space where the hypothesized constraint lives, and its projected
performance ($\sim$148 PFLOPS per chip for matrix operations) enables Benford analysis
across datasets at scales not previously attempted.

%% ============================================================
\section{Conclusion}
%% ============================================================

We have proposed that Benford's logarithmic distribution of leading digits is not an
emergent statistical regularity but the single axiom underlying physical reality ---
the constraint that all emergent systems must satisfy in order to exist. Light's
constancy, the foundation of modern physics, is not a separate axiom but the purest
physical expression of this constraint: with zero mass, a photon has zero entropy
and therefore zero deviation from the logarithmic baseline. Einstein discovered the
most visible consequence of the axiom. The axiom itself is Benford's distribution.

This framework generates specific, testable predictions: that quantum equations,
when measured against the Benford baseline, will show characterizable deviation
patterns that resolve at the decoherence boundary; and that gravitational equations,
when brought to the constraint, will satisfy it in a manner distinct from the other
three forces --- consistent with gravity being the self-regulatory expression of the
underlying logarithmic constraint itself.

If confirmed, this framework would provide an empirical anchor for unification
physics --- something string theory has never achieved --- and would suggest that the
question of what existed before the Big Bang is answerable: the logarithmic
constraint that made the Big Bang possible.

The formula has been on the page since 1881. The question is whether we have been
reading it correctly.

%% ============================================================
\begin{thebibliography}{30}
%% ============================================================

\bibitem{newcomb1881}
S.~Newcomb, ``Note on the Frequency of Use of the Different Digits in Natural
Numbers,'' \emph{American Journal of Mathematics}, vol.~4, no.~1, pp.~39--40, 1881.

\bibitem{benford1938}
F.~Benford, ``The Law of Anomalous Numbers,'' \emph{Proceedings of the American
Philosophical Society}, vol.~78, no.~4, pp.~551--572, 1938.

\bibitem{burke1991}
J.~Burke and E.~Kincanon, ``Benford's Law and Physical Constants: The
Distribution of Initial Digits,'' \emph{American Journal of Physics}, vol.~59, no.~10,
pp.~952--954, 1991.

\bibitem{ni2008}
D.~Ni and Z.~Ren, ``Benford's Law and Half-Lives of Unstable Nuclei,'' \emph{European
Physical Journal A}, vol.~38, pp.~251--255, 2008.

\bibitem{shao2009}
L.~Shao and B.-Q.~Ma, ``First Digit Distribution of Hadron Full Width,'' \emph{Modern
Physics Letters A}, vol.~24, no.~30, pp.~2465--2474, 2009.

\bibitem{shao2010}
L.~Shao and B.-Q.~Ma, ``The Significant Digit Law in Statistical Physics,''
\emph{Physica A}, vol.~389, no.~16, pp.~3109--3116, 2010.

\bibitem{ralchenko2024}
Y.~Ralchenko and J.-C.~Pain, ``Benford's Law in Atomic Spectra and Opacity
Databases,'' \emph{Journal of Quantitative Spectroscopy and Radiative Transfer}, vol.~322,
109010, 2024.

\bibitem{alexopoulos2014}
T.~Alexopoulos and S.~Leontsinis, ``Benford's Law in Astronomy,'' \emph{Journal of
Astrophysics and Astronomy}, vol.~35, pp.~639--648, 2014.

\bibitem{pinkham1961}
R.~S.~Pinkham, ``On the Distribution of First Significant Digits,'' \emph{Annals of
Mathematical Statistics}, vol.~32, no.~4, pp.~1223--1230, 1961.

\bibitem{hill1995base}
T.~P.~Hill, ``Base-Invariance Implies Benford's Law,'' \emph{Proceedings of the
American Mathematical Society}, vol.~123, no.~3, pp.~887--895, 1995.

\bibitem{hill1995statistical}
T.~P.~Hill, ``A Statistical Derivation of the Significant-Digit Law,''
\emph{Statistical Science}, vol.~10, no.~4, pp.~354--363, 1995.

\bibitem{kafri2009}
O.~Kafri, ``Entropy Principle in Direct Derivation of Benford's Law,''
arXiv:0901.3047, 2009.

\bibitem{lemons2019}
D.~S.~Lemons, ``Thermodynamics of Benford's First Digit Law,'' \emph{American Journal
of Physics}, vol.~87, no.~10, pp.~787--790, 2019.

\bibitem{burgos2021}
A.~Burgos and A.~Santos, ``The Newcomb-Benford Law: Scale Invariance and a
Simple Markov Process Based on It,'' \emph{American Journal of Physics}, vol.~89, no.~9,
pp.~851--861, 2021.

\bibitem{berger2011}
A.~Berger and T.~P.~Hill, ``A Basic Theory of Benford's Law,'' \emph{Probability
Surveys}, vol.~8, pp.~1--126, 2011.

\bibitem{shannon1948}
C.~E.~Shannon, ``A Mathematical Theory of Communication,'' \emph{Bell System Technical
Journal}, vol.~27, pp.~379--423 and 623--656, 1948.

\bibitem{boltzmann1877}
L.~Boltzmann, ``\"{U}ber die Beziehung zwischen dem zweiten Hauptsatze der
mechanischen W\"{a}rmetheorie und der Wahrscheinlichkeitsrechnung,'' \emph{Wiener Berichte},
vol.~76, pp.~373--435, 1877.

\bibitem{jaynes1957}
E.~T.~Jaynes, ``Information Theory and Statistical Mechanics,'' \emph{Physical Review},
vol.~106, no.~4, pp.~620--630, 1957.

\bibitem{wilson1971}
K.~G.~Wilson, ``Renormalization Group and Critical Phenomena.\ I.,'' \emph{Physical
Review B}, vol.~4, pp.~3174--3183, 1971.

\bibitem{verlinde2011}
E.~Verlinde, ``On the Origin of Gravity and the Laws of Newton,'' \emph{Journal of
High Energy Physics}, 2011, 29, 2011.

\bibitem{jacobson1995}
T.~Jacobson, ``Thermodynamics of Spacetime: The Einstein Equation of State,''
\emph{Physical Review Letters}, vol.~75, pp.~1260--1263, 1995.

\bibitem{sambridge2010}
M.~Sambridge, H.~Tkalcic, and A.~Jackson, ``Benford's Law in the Natural
Sciences,'' \emph{Geophysical Research Letters}, vol.~37, L22301, 2010.

\bibitem{sende2011}
A.~Sen(De) and U.~Sen, ``Benford's Law Detects Quantum Phase Transitions
Similarly as Earthquakes,'' \emph{Europhysics Letters}, vol.~95, 50008, 2011.

\bibitem{rane2014}
A.~D.~Rane, U.~Mishra, A.~Biswas, A.~Sen(De), and U.~Sen, ``Benford's Law
Gives Better Scaling Exponents in Phase Transitions of Quantum XY Models,'' \emph{Physical
Review E}, vol.~90, 022144, 2014.

\bibitem{riner2025}
C.~Riner, ``N-Radix: Wavelength-Division Ternary Optical Accelerator,'' 2025.
Open source: \url{github.com/jackwayne234/-wavelength-ternary-optical-computer}

\bibitem{bousso2002}
R.~Bousso, ``The Holographic Principle,'' \emph{Reviews of Modern Physics}, vol.~74,
pp.~825--874, 2002.

\bibitem{hawking1975}
S.~W.~Hawking, ``Particle Creation by Black Holes,'' \emph{Communications in
Mathematical Physics}, vol.~43, pp.~199--220, 1975.

\bibitem{calabrese2004}
P.~Calabrese and J.~Cardy, ``Entanglement Entropy and Quantum Field Theory,''
\emph{Journal of Statistical Mechanics}, P06002, 2004.

\bibitem{ryu2006}
S.~Ryu and T.~Takayanagi, ``Holographic Derivation of Entanglement Entropy from
AdS/CFT,'' \emph{Physical Review Letters}, vol.~96, 181602, 2006.

\bibitem{maldacena1998}
J.~Maldacena, ``The Large N Limit of Superconformal Field Theories and
Supergravity,'' \emph{Advances in Theoretical and Mathematical Physics}, vol.~2,
pp.~231--252, 1998.

\end{thebibliography}

\vspace{1em}
\noindent\textit{Correspondence: \texttt{chrisriner45@gmail.com}}\\
\noindent\textit{N-Radix Project: \url{github.com/jackwayne234/-wavelength-ternary-optical-computer}}

\end{document}
